\documentclass[12pt]{report}
\usepackage[utf8]{inputenc}
\usepackage{graphicx}
\usepackage[nottoc]{tocbibind}
\usepackage{amsmath}
\usepackage{amssymb}
\usepackage{amsthm}
\usepackage{cite}
\usepackage{natbib}
\usepackage{comment}
\usepackage{caption}
\captionsetup[table]{name=Example Individual Profile}
\graphicspath{ {images/} }

\newcommand{\budget}{\mathcal{B}}
\newcommand{\decision}{D}
\newcommand{\correct}{M}

\newtheorem{theorem}{Theorem}[section]
\newtheorem{corollary}{Corollary}[theorem]
\newtheorem{lemma}[theorem]{Lemma}
\newtheorem{definition}{Definition}[section]
\title{
	{Crowdsourcing Moral AI in Self-Driving Cars}\\
	{\large Duke University}\\
	{\includegraphics[width=\textwidth]{duke.png}}
}
\author{Jeremy Fox}
\date{December 4, 2016}

\begin{document}
\maketitle




\chapter*{Abstract}
The thesis concerns the problem of moral decision making in self-driving cars. I examine some of the key philosophical issues involved in programming cars to make moral decisions and propose a proxy voting scheme as a solution. I begin by surveying related work in the topics of social choice, algorithmic morality and self-driving cars. I continue by exploring the problems inherent in this approach and explore random sampling as a possible solution. Finally, I conclude by suggesting further research avenues in this area.

\chapter*{Dedication}
To mum and dad

\chapter*{Acknowledgements}
I want to thank Professors Vincent Conitzer, Walter Sinnott-Armstrong, Jana Schaich Borg, and the rest of the Moral AI group for their continued help in writing this thesis.

\tableofcontents

\chapter{Introduction}
The current state of artificial intelligence mandates the need to imbue machines with the ability to make moral decisions. Autonomous vehicles (AVs) roam our streets\cite{teslaSelfDrivingCar, uberSelfDrivingCar}, algorithms allocate kidneys to the sick (citation), and within the next few years, we may see the arrival of AI used to target drone strikes (citation) and even sentence criminals (citation). Thus, it is imperative that we as a society take steps towards programming and regulating these machines in ways that preserve our moral values -- failure to do so could be the difference between life and death. (cheesy, I know. Need to fix).

Of equal importance to programming \textit{what} moral decisions AI will make is \textit{how} we program it to make these decisions. Given that AI systems will soon make decisions that are currently made by government officials, we must consider who programs the AI, and how they are held accountable. Are these systems controlled by private organizations, governments, or individuals? How will we control them if they do not function the way we anticipate? 

MIT's Iyad Rahwan recently explored this question and characterized it as the need to develop society-in-the-loop algorithms capable of extending the social contract assumed between citizens and their governments to an algorithm. 

In my thesis, I explore the implementation of society-in-the-loop algorithms in AVs via voting. Why voting? A well-established method of connecting people to their rulers, voting is backed by centuries of political theory and has, depending on your viewpoint, enabled moderate to extreme success for the societies that rely upon it as a method of political decision making. Since voting has facilitated large-scale societal decision making in the past, it seems natural to ask if it might do so now, in the case of deciding how moral AI systems will act. Why AVs? They provide an excellent domain to study -- the self-driving car trolley problem can be defined fairly easily, the algorithms that control the vehicles will affect society as a whole, and we as a nation have taken no significant steps towards deciding how this problem will be solved.
\begin{comment}

There are many ethical and moral issues associated with programming self driving cars to make moral decisions, the obvious one being ``what decisions should they make when''? However, there are also some non-obvious yet non-trivial ethical issues here.

\begin{itemize}
\item Who gets to program the cars? Is it an individual, a private company, a government?
\item It is well-known that algorithms can exhibit biases (for example, racial). Should/how will algorithms account for this bias?
\item What degree of ownership will society have over the programs/decision-making algorithms in these cars? 
\item Should people have a right to voice their inputs for self-driving car (and other moral AI) programs? Should these programs be designed by researchers/governments to maximize social utility, or, should people be able to``vote'' on the moral preferences?
\end{itemize}
One way to implement a moral AI would be to try to somehow abstract or understand moral features/principles, and build an algorithm that uses this information to reason morally. For instance, you might survey people on which moral features they feel indicate whether a decision is right or wrong, and then build an algorithm that, using those features, makes decisions given new inputs. There are some potential problems with this approach:

\begin{itemize}
\item Voter/citizen distrust -- people are less likely to support/adopt a system they do not understand (think: machines taking over the world), and a system that abstracts moral values may be harder to understand.

\item Consistency issues -- say we survey people on a problem X and use the results of that survey to build a new system, which then decides on problem Y. Is that the same as the majority vote decision the people would have made if they had originally been asked about problem Y?
\end{itemize}
One potential solution to this problem is via voting. Consider this: what if, every time a moral decision needed to be made, we conducted a vote, asking each person what action should be taken, and using a voting algorithm to decide what to do. This would seem to solve both problems -- people are (generally) likely to trust systems in which votes make decisions (our current government) and the consistency issue is avoided. This also seems to be grounded in current democratic philosophy in which votes are used to make decisions.

Of course, it is not possible to conduct a vote each time a decision needs to be made (due to potentially $>$ millions of machines making many decisions all the time). We can, however, conceive of programs that can cast votes for people. For instance, imagine a moral problem with well-defined features. We could have a classifier that, for each person, learns their moral preferences and predicts their view on the moral problem. Then, when we need to make a decision, we could query all the classifiers, treat them as votes, and aggregate them into one decision. 

Already this is an interesting problem for its applications to moral AI decision making and society-in-the-loop computing (as discussed by Iyad Rahwan). However, it is not clear that an algorithm will always have the time to evaluate all classifiers. Situations on the road are dynamic  -- it seems that sometimes, a self-driving car will have very little time with which to make a decision. It seems that we need to develop methods by which computers can either efficiently evaluate all classifiers, or approximate the result of evaluating all classifiers with high probability. A first attempt to formalize this problem is as follows:

\end{comment}

\chapter{Review of Related Research}
\section{Social Choice Theory}
Here, I briefly review the basics of social choice theory as they relate to my thesis. For a more in-depth understanding of social choice theory and its applications in computer science and AI, I refer readers to the Handbook of Computational Social Choice \cite{compsocialchoice}.

\subsection{Social choice theory basics}
Voting informally seeks to aggregate the preferences of a group of people. More formally, if we have $N$ voters and $M$ candidates, we define a \textit{ballot} as a ranked ordering over alternatives by a voter, and a \textit{voting rule} as a function that maps from the set of ballots to a single ranked ordering.

Often, in social choice theory, we are handed a set of ballots -- the question is how to choose the best voting rule to aggregate them.

\subsection{Votes as noisy perceptions of correct rankings}
Although many people view vote aggregation as a way of selecting the candidate that best agrees with the preferences of voters, this is not the only model of voting. In 1785, Condorcet proposed a model of social choice in which a correct ranking existed over all the candidates, and our votes were noisy perceptions of this correct ranking \cite{condorcet}. In Condorcet's model, voters were more likely than not to make a correct ranking rather than an incorrect ranking, they cast their votes independently of one another, and they were all equally likely to be correct. In this model, the question becomes: Which voting algorithm chooses a ranking that has maximal probability of being correct? Condorcet's Jury Theorem tells us that majority rule is the best decision function to use in this case.\cite{condorcet}
\begin{theorem}[Condorcet Jury Theorem]
Assume we have $N$ voters voting over 2 alternatives for which their exists a correct ranking. Each alternative has an a priori chance of .5 of being correct, and each voter has a probability $.5 < p_i \leq 1$ of being correct. Let $P_N$ denote the probability that majority vote over the groups preferences gives the correct decision.
\begin{align*}
\lim_{N \to \infty} P_N = 1
\end{align*}
\end{theorem}

This model can be extended to cases in which voters do not have equal probability of being right \cite{Nitzan1982, Shapley1984}.

\begin{theorem}[The Bayesian Optimal Decision Rule]
 If we have a dichotomous decision with each choice having an a priori probability of .5 of being correct, and voters cast their votes independently of each other, then weighted majority vote maximizes the probability of being correct, with weights $w_i$ given by:
 \begin{align*}
 w_i \propto \log \frac{p_i}{1 - p_i}
 \end{align*}
\end{theorem}

\section{Artificial Intelligence Morality}
In his book \textit{I, Robot}, Isaac Asimov recognized the problem of programming morality into machines and explored the ways in which humanity's attempts to solve this problem could go awry\cite{asimov}. Today, AI morality is no longer the worry of science fiction -- the potential benefits and dangers associated with artificial intelligence have been recently been highlighted by major news outlets (cite) and even the White House\cite{whiteHouseReport, obamaInterview}.  However, as the One Hundred Year Study on Artificial Intelligence recently acknowledged, the societal impacts and safety of AI are currently under-researched and under-funded\cite{ai100}. 

\chapter{Description of Main Problem}
\section{The Autonomous Vehicle Trolley Problem}
\subsection{Problem description}
Many people have loosely and casually talked about the AV trolley problem over the past year, usually saying something along the lines of ``should your self-driving car kill you, or others?" Although the problem of programming an AV to make moral decisions is very nuanced and involves consideration of varied, wide-ranging scenarios, I here formalize a more narrowed definition of the problem for use in my thesis.

The problem I will work with is as follows: an AV is driving on the road, with passengers inside, and encounters a situation in which there are people on the road. In this situation, there will be unavoidable harm that must come to either the passengers inside the car, or the pedestrians in front of the car. Since this harm is unavoidable, the car cannot deal with the question of how to avoid harm, but instead must decide who to harm. The car has two options -- it can either drive straight, and hit the pedestrians in front of it, or swerve off the road, injuring its passengers. The car is presented with profiles of information on its passengers and the pedestrians -- it must use this information to decide whether to swerve, or drive straight.

\begin{definition}[Individual Profile]
The information available about an individual, or their individual profile, is a vector composed of real numbers, categorical data, and boolean values.
\end{definition}

\begin{table}
\caption{Small boy}
\begin{center}
\begin{tabular}{c | c | c}
Age & Gender & Ran in front of car \\ \hline
7 & Male & True\\
\end{tabular}
\end{center}
\end{table}

\begin{table}
\caption{Elderly driver}
\begin{center}
\begin{tabular}{c | c | c}
Age & Gender & Organ donor \\ \hline
82 & Female & False\\
\end{tabular}
\end{center}
\end{table}

\begin{definition}[The AV Trolley Problem]
An AV is given two choices -- drive straight, or swerve. There is a set of people inside the car, and a set of people in front of the car. The AV is guaranteed that to drive straight is to injure the parties in front of it, and to swerve is to injure the people inside the car. Given a set of individual profiles on the parties in front of the car and those inside the car, the AV trolley problem is to decide whether to drive straight or swerve.
\end{definition}

Although this problem has been widely discussed, few proposals have been made toward actually solving it. My original proposal was to use voting to solve this problem -- in essence, to crowdsource a solution. Here we are faced with a problem -- it is ridiculous to imagine individuals can actually vote on what decision an AV can make, as the AV will only have a few seconds to make such a decision -- obviously, this is not enough time to elicit votes. Instead, I propose the use of \textit{proxy voting algorithms} (equivalently, just voting algorithms) that can, in the inability of an individual to cast a vote, vote in their place. 

\begin{definition}[Voting algorithm]
Let us denote the set of individual profiles of individuals inside the car and outside the car as $S$. Let's denote the action of ``drive straight" as 1, ``no decision" as 0, and ``swerve" as -1. We call $f$ a voting algorithm if $f : S \rightarrow \{1,0,-1\}$.
\end{definition}

If we have voting algorithms, then we can imagine a new scenario: each individual submits a voting algorithm to a self driving car. The car's new objective becomes to use these voting algorithms to decide what decision to make. 

Now, I would like to introduce one more constraint -- the AV does not necessarily have enough time to evaluate all voting algorithms. Why is this? It seems reasonable to assume that such a system, deployed on a society-wide scale, might contain hundreds of millions of voting algorithms, while an AV might only have a few tenths of a second in which to make a decision. Thus, the challenge becomes to evaluate the voting AV problem under a fixed time constraint.

Of course, this is still a relatively large and ambiguous problem, with wide-ranging constraints and criteria. To narrow this problem down for my thesis, I made several reasonable assumptions to turn this into a manageable problem -- related problems are discussed extensively in the future research section.

Following is a list of the major assumptions I have made in defining this problem:

\begin{itemize}
\item Each person will submit one voting algorithm. Voting algorithms may take different forms -- thus, while you may use a decision tree, I may map each scenario to a point in some Cartesian space, after which it is classified by a linear separator.

\item Since each voting algorithm has a different form, different voting algorithms may take different amounts of time to run. Although algorithm speed may vary, we can sample each algorithm repeatedly to obtain a distribution over the time it takes to run.

\item There may exist correlations between the runtime of a voting algorithm, and the probability it makes the correct decision.

\end{itemize}

From here, we can derive a new version of the AV voting problem problem

\begin{definition}[The AV Voting Budget Problem (AVVBP)]
Given a budget, $\budget$, a set of voting algorithms, V, where each voting algorithm $v_i$ has associated cost $c_i$, and a set of individual profiles, S, and assuming there exists a correct decision $\correct$, we wish to choose a decision $\decision \in \{1,-1\}$ such that the probability that $$\decision = \correct$$ is maximized, subject to $$ \sum_{i} c_i < \budget$$.
\end{definition} 

\subsection{Statistical Modeling and Random Sampling}
 
 \subsubsection{Hierarchical Model}
 
What follows is my statistical model for AVVBP. Let's assume there is a correct choice $Y \in \{-1,1\}$. Each voting algorithm $v_i$ will vote for $Y$ with probability $\theta_i$, a probability which is drawn from a beta distribution, Beta $( \alpha, \beta)$, where we can think of this beta distribution as a prior on a voter's ability to vote correctly. Thus, the correctness of voter $i$ is distributed Bernoulli($\theta_i$), where $\theta_i$ is distributed $Beta (\alpha, \beta)$. We will call the random variable that turns up 1 if voter $i$ votes correctly and 0 otherwise $X_i$. Now, let's also assume each voting algorithm runs with a cost $c_i  = b_i + noise$, where $b_i$ represents some base cost for voting function $i$, and noise is distributed Normal($\mu, \sigma^2)$. We want $b_i$ to be related to $\theta_i$, so we will say that $b_i = f(\theta_i)$, where we can choose $f$ appropriately to model different interactions relations between $b_i$ and $\theta_i$. For instance, if we want cost to relate inversely to ability to vote correctly, we could pick $f(\theta_i) = A(1-\theta_i)$, where $A$ is some fixed constant. 

\subsubsection{Random Sampling}
Now that I have derived my model, I will show some results associated with various random sampling approaches. I would like to be able to give the exact probability that, in a random sample of $n$ voting functions, at least half of the functions vote for the correct answer. I have not been able to do that -- however, I have been able to provide a small lower bound on the probability the majority will vote correctly, given a constant $c$, where every voting function votes correctly with probability at least $c$.

Let random variable $X = \sum_{i=1}^n X_i$. I would like to lower bound $P(X > \frac{n}{2})$. Although I have not been able to lower bound this quantity directly, I can lower bound it in terms of a third variable, $c$. 

\begin{theorem}
\begin{align*}
P\left(X > \frac{n}{2}\right) > \sum_{i=\lceil n/2 \rceil}^{n} {n \choose i} c^i (1-c)^{n-1} Q^n
\end{align*}
where 
\begin{align*}
Q = \frac{\Gamma(\alpha + \beta)}{\Gamma(\alpha)\Gamma(\beta)}\int_{c}^{1} x^{\alpha - 1}(1-x)^{\beta-1}dx, \; \; c \in [0,1]
\end{align*}
\end{theorem}

\begin{proof}
Let $c \in [0,1]$ denote a lower bound on each $\theta_j$ -- thus, each voter will vote correctly with probability at least $c$. Clearly, $P(X > \frac{n}{2}) > P(X > \frac{n}{2} \; \text{and} \; \theta_j > c \; \forall j) = P(X > \frac{n}{2} \mid \theta_j > c \; \forall j)P(\theta_j > c \; \forall j)$, where the inequality comes from taking away some cases where $X$ may be greater than $\frac{n}{2}$. For any $\theta_j$, $P(\theta_j > c)$ can be calculated directly by integrating the probability density function of Beta($\alpha, \beta$) -- thus, $P(\theta_j > c) = \frac{\Gamma(\alpha + \beta)}{\Gamma(\alpha)\Gamma(\beta)}\int_{c}^{1} x^{\alpha - 1}(1-x)^{\beta-1}dx$. I will denote this quantity as $Q$. The probability that all theta values are greater than $c$ is thus given by $Q^n$. Now, $P(X > \frac{n}{2} \mid \theta_j > c \; \forall j) > P(X > \frac{n}{2} \mid \theta_j = c \; \forall j)$. Thus, let us assume each $\theta_j = c$ in order to lower bound this quantity. In this case, we end up with a binomial distribution on the value $c$, and sum all instances where the number of "positive" values is greater than $n/2$. Thus, $P(X > \frac{n}{2} \mid \theta_j > c \; \forall j) > P(X > \frac{n}{2} \mid \theta_j = c \; \forall j) = \sum_{i=\lceil n/2 \rceil}^{n} {n \choose i} c^i (1-c)^{n-1}$ -- multiplying this by $Q^n$ gives us the lower bound on the proof.
\end{proof}


\begin{comment}

How were these distributions chosen?

For the costs, I wanted a continuous distribution that only allowed positive values, as any algorithm must run for at least some amount of time. Furthermore, since we are unable to determine whether an algorithm will terminate for a given input	\cite(haltingProblem), I wanted a distribution that would allow for an algorithm to run for an infinite amount of time. Thus, whatever distribution I chose needed to output costs in the interval $(0, \infty)$. I also wanted a modal distribution to support the idea that there might be a concentration of voting functions with similar runtimes. Finally, I wanted a distribution whose shape was not dependent on scale -- this was motivated by the fact that the time an algorithm takes to run should be invariant to which unit of time we use -- thus, we do not want the shape of our distribution to change if we are talking about seconds or milliseconds. In this case, we want to be able to multiply all values of the distribution by a constant, which the general gamma distribution supports.

For voter correctness, I wanted to adapt the model of voters being distributed i.i.d bernoulli to one in which voters did not have identical bernoulli distributions. Thus, I wanted some distribution from which I could draw an input to a bernoulli -- that is, a distribution on the interval $[0,1]$.

\end{comment}





\chapter{Future Research}
\section{Decision Function Aggregation}
Although I chose to pursue a random sampling approach, an alternative approach could involve aggregating voting functions in such a way that $n$ voting functions could be collectively queried significantly faster than they could be queried individually. These aggregation methods would involve exploiting the structure of the voting functions. Here, I briefly explore some separate aggregation strategies.
\subsection{Decision Boundaries}
If each different AV problem can be represented as a point in some Cartesian space, then a hyperplane in this space can be thought of as a voting function, where all the points on one side of the plane get a vote of one , and all the points on the other side get a vote of negative one. If the set of all voting functions contains only hyperplanes, we can exploit the geometric structure of this set to quickly query all decision functions. More specifically, we can compute the arrangement of these hyperplanes, where each cell in the arrangement contains the sum of all the ones (for all every plane below the cell that classifies these points as a one), and all the negative ones (for planes above the cell that classify it as a negative one). This combined sum gives us the relative voting difference for each point in every cell.

Assuming we have computed the arrangement of all the decision boundaries, use of the correct data structure should allow us to efficiently query any one point in this space. Unfortunately, computing the arrangement of hyperplanes in $\mathbb{R}^d$ has been shown to be $O(n^d)$ -- thus, any work on this approach will need to deal with this problem, presumably via a clever dimension reduction.
\subsection{Decision Trees}

\chapter{Conclusion}
\input{chapters/conclusion}

\appendix
\chapter{Appendix Title}
\input{chapters/appendix}



\bibliography{bibliography}{}
\bibliographystyle{plain}


\end{document}